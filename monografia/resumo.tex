\begin{resumo}
Com o aumento da demanda por modelos 3D em �reas como jogos eletr�nicos, a gera��o procedural se faz cada vez mais necess�ria, para reduzir custos e tempo de desenvolvimento. Atrav�s de t�cnicas procedurais, � poss�vel criar um grande n�mero de modelos com a varia��o de alguns poucos par�metros dos algoritmos respons�veis pela gera��o. Por�m, t�cnicas procedurais se caracterizam por uma baixa intera��o do usu�rio, gerando assim modelos nem sempre expressivos. Este trabalho � focado na gera��o de terrenos proceduralmente e aqui ser� proposto um arcabou�o que insira no processo uma maior participa��o do usu�rio, com o objetivo de criar modelos que traduzam melhor a sua espectativa. T�cnicas procedurais j� difundidas s�o exploradas e mescladas com inser��o de terrenos e modelos por parte do usu�rio.

    \textbf{Palavras-chave:} Gera��o procedural de conte�do, terreno.
\end{resumo}
