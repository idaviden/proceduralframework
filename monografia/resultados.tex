\chapter{RESULTADOS E DISCUSS�O}
Os algoritmos de ru�do de perlin, e fractal plasma foram implementados, bem como a leitura de imagens representando mapas de altura. Para uma melhor visualiza��o, o arcabou�o possui uma movimenta��o b�sica com o mouse e teclado.


As Figuras \ref{fig:tela2} e \ref{fig:tela3}, e \ref{fig:tela6} e \ref{fig:tela7} mostram os terrenos gerados variando o n�mero de itera��es do ru�do de Perlin e o n�mero de terrenos vizinhos exibidos.

\begin{figure}[H]
	\center{\includegraphics[width=0.5\linewidth]{img/caps/2.png}}
	\caption{\label{fig:tela2} Tela com o terreno gerado.}
\end{figure}

\begin{figure}[H]
	\center{\includegraphics[width=0.5\linewidth]{img/caps/3.png}}
	\caption{\label{fig:tela3} Tela com o terreno gerado (exibi��o em \emph{wireframes}).}
\end{figure}

\begin{figure}[H]
	\center{\includegraphics[width=0.5\linewidth]{img/caps/6.png}}
	\caption{\label{fig:tela6} Tela com o terreno gerado.}
\end{figure}

\begin{figure}[H]
	\center{\includegraphics[width=0.5\linewidth]{img/caps/7.png}}
	\caption{\label{fig:tela7} Tela com o terreno gerado (exibi��o em \emph{wireframes}).}
\end{figure}

As Figuras \ref{fig:tela4} e \ref{fig:tela5} mostram um terreno gerado proceduralmente e o mapa de altura exibido na Figura \ref{fig:mapaaltura} inserido no arcabou�o (mostrado no arcabou�o em um tom cinza mais escuro).

\begin{figure}[H]
	\center{\includegraphics[width=0.25\linewidth]{img/heightmap.png}}
	\caption{\label{fig:mapaaltura} Mapa de altura.}
\end{figure}

\begin{figure}[H]
	\center{\includegraphics[width=0.5\linewidth]{img/caps/4.png}}
	\caption{\label{fig:tela4} Tela com o terreno gerado e um mapa de altura inserido.}
\end{figure}

\begin{figure}[H]
	\center{\includegraphics[width=0.5\linewidth]{img/caps/5.png}}
	\caption{\label{fig:tela5} Tela com o terreno gerado e um mapa de altura inserido (exibi��o em \emph{wireframes}).}
\end{figure}


Alguns testes foram feitos, para avaliar a varia��o de frames por segundo (\sigla{FPS}{Frames por segundo}) com a altera��o de alguns par�metros. Eles foram executados em um \emph{Athlon64 3500}, com 2GB de mem�ria \emph{RAM} e placa de v�deo GeForce6600 com 64MB de mem�ria.

A Se��o \ref{teste_geracao} mostra um teste com a gera��o de 100 terrenos proceduralmente.

As Se��es \ref{teste_variando_octaves} e \ref{teste_variando_vizinhos} apresentam testes que consistiam em um v�o da c�mera pelo terreno durante 60 segundos, em um trajeto constante para todos os testes.

\section{Teste 1}
\label{teste_geracao}
No primeiro teste, foi medido o tempo gasto com a gera��o de terrenos. Para cada n�mero de octaves (4, 16 e 32), foi gerado 100 terrenos, e medido o tempo gasto.

Nas Tabelas \ref{tabela:geracao1} e \ref{tabela:geracao2} � apresentado os tempos das chamadas �s fun��es respons�veis pela gera��o dos terrenos, considerando o algoritmo de ru�do de Perlin. Abaixo uma explica��o sobre cada fun��o:

\begin{itemize}
	\item FillHeightmap: Constr�i o terreno, determinando a altura e posi��o de cada v�rtice e armazena em um vetor de \emph{float}. Sua complexidade � \emph{O(n*m*c)}, onde \emph{n} � a largura do terreno, \emph{m} o comprimento e \emph{c} � o n�mero de octaves.
	\item CopyHeightmap: Copia o vetor, constru�do na fun��o \emph{FillHeightmap} para um novo vetor VBO. Sua complexidade � \emph{O(n*m)}, onde \emph{n} � a largura do terreno e \emph{m} o comprimento.
	\item BuildVBOs: Modifica os ponteiros para o desenho dos vetores VBO. � independente do tamanho do vetor (\emph{O(1)}).
\end{itemize}

\begin{table}[H]
	\begin{center}
		\begin{tabular}{|c|c|c|c|c|c|c|}
			\hline
			\multirow{2}{*}{Octaves} & \multicolumn{3}{|c|}{FillHeightMap} & \multicolumn{3}{|c|}{CopyHeightMap} \\
			\hline
			 & \scriptsize M�x. & \scriptsize M�n. & \scriptsize M�dia & \scriptsize M�x. & \scriptsize M�n. & \scriptsize M�dia \\
			\hline
			4 & 0.0536568 & 0.0410248 & 0.044377716 & 0.0866761 & 0.0737747 & 0.078294278 \\
			\hline
			16 & 0.259341 & 0.164484 & 0.16980303 & 0.0865546 & 0.0710867 & 0.073424774 \\
			\hline
			32 & 0.45034 & 0.331501 & 0.3405685 & 0.0778093 & 0.0708154 & 0.073012029 \\
			\hline
			
		\end{tabular}
		\caption{Tempos da gera��o dos terrenos (em segundos)}
		\label{tabela:geracao1}
	\end{center}
\end{table}

\begin{table}[H]
	\begin{center}
		\begin{tabular}{|c|c|c|c|c|c|c|}
			\hline
			\multirow{2}{*}{Octaves} & \multicolumn{3}{|c|}{BuildVBOs} & \multicolumn{3}{|c|}{Total}  \\
			\hline
			 & \scriptsize M�x. & \scriptsize M�n. & \scriptsize M�dia & \scriptsize M�x. & \scriptsize M�n. & \scriptsize M�dia \\
			\hline
			4 & 0.0161562 & 0.00783591 & 0.0107964372 & 0.14988 & 0.126432 & 0.13364766 \\
			\hline
			16 & 0.0407795 & 0.0081773 & 0.0114127442 & 0.34595 & 0.246974 & 0.25501801 \\
			\hline
			32 & 0.0129804 & 0.00754481 & 0.0093645482 & 0.531098 & 0.412121 & 0.42311833 \\
			\hline
		\end{tabular}
		\caption{Tempos da gera��o dos terrenos (em segundos)}
		\label{tabela:geracao2}
	\end{center}
\end{table}



A Tabela \ref{tabela:geracao3} apresenta os percentuais de tempo de cada chamada da fun��o em rela��o ao tempo m�dio total gasto com a gera��o de um terreno. A Figura \ref{fig:geracao3} mostra um gr�fico com esses dados.

\begin{table}[H]
	\begin{center}
		\begin{tabular}{|c|c|c|c|c|}
			\hline
			\emph{Octaves} & FillHeightMap & CopyHeightMap & BuildVBOs & Total  \\
			\hline
			4 & 32,2\% & 58,6\% & 9,2\% & 100\% \\
			\hline
			16 & 66,6\% & 28,8\% & 4,6\% & 100\% \\
			\hline
			32 & 80,5\% & 17,3\% & 2,2\% & 100\% \\
			\hline
		\end{tabular}
		\caption{Porcentagem dos tempos m�dios de cada chamada em rela��o ao tempo total m�dio de gera��o}
		\label{tabela:geracao3}
	\end{center}
\end{table}


\begin{figure}[H]
	\center{\includegraphics[width=0.5\linewidth]{testes/testes.png}}
	\caption{\label{fig:geracao3} Gr�fico com a porcentagem dos tempos m�dios de cada chamada em rela��o ao tempo total m�dio de gera��o}
\end{figure}

Como podemos observar, o tempo gasto com a c�pia dos dados para a nova estrutura de dados (\emph{CopyHeightMap}) gasta um tempo consider�vel nas tr�s medi��es (com 4, 16 e 32 octaves), algo a ser considerado em trabalhos futuros.

\section{Teste 2}
\label{teste_variando_octaves}
\input{teste_variando_octaves}


\section{Teste 3}
\label{teste_variando_vizinhos}
O pr�ximo teste (Figura \ref{fig:teste2}) mostra o impacto variando o n�mero de terrenos vizinhos. Como era de se esperar, quanto maior o n�mero de vizinhos, menor o FPS.

\begin{figure}[H]
	\center{\includegraphics[width=0.5\linewidth]{img/graficos/teste2/teste2.png}}
	\caption{\label{fig:teste2} Teste variando o n�mero de terrenos vizinhos, e o n�mero de octaves fixo.}
\end{figure}

