\chapter{CONCLUS�ES E TRABALHOS FUTUROS}
O trabalho desenvolvido em POC I teve como objetivo principal estabelecer uma base para ser desenvolvida posteriormente em POC II, e tamb�m como uma forma de aprendizado de t�cnicas procedurais.

Um ponto a ser otimizado � a c�pia dos v�rtices para as estruturas VBO. Dessa forma, ser� poss�vel diminuir o tempo total gasto com a gera��o procedural dos terrenos, e suavizar as transi��es entre terrenos.

Um ponto que s� foi iniciado em POC I � a gera��o de terrenos esf�ricos e deve ser abordado em POC II. Al�m disso, o uso de texturas tamb�m ser� aprofundado, possivelmente com o uso de sombras pr�-calculadas \cite{slopelighting}, pois atualmente, os terrenos podem ser facilmente confundidos com nuvens, �gua parada, ou qualquer outro fen�meno natural. Outra quest�o a ser desenvolvida em POC II � uma interface gr�fica mais atrativa, e uma forma eficiente de se armazenar as informa��es inseridas pelo usu�rio (possivelmente em arquivos \sigla{XML}{Extensible Markup Language}).


