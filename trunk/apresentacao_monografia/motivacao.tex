\section{Motiva��o} 
\begin{frame}\frametitle{Motiva��o} 

\begin{itemize}
	\item Atualmente, h� uma necessidade de se criar modelos 3D cada vez maiores e com grande n�vel de detalhe.
	\item Por�m, quanto maior e mais detalhado o modelo, mais tempo ter� que ser gasto por um modelador para faz�-lo.
	\item A� entra a gera��o procedural...
\end{itemize}
	
\end{frame}

\subsection{Gera��o Procedural}
\begin{frame}\frametitle{O que � gera��o procedural?} 

\begin{itemize}
	\item Gera��o procedural � um termo gen�rico para descrever algoritmos que determinam caracter�sticas de efeitos ou modelos.
	\item H� diversos tipos de t�cnicas e algoritmos, cada um aplicado a uma determinada �rea:
	\begin{itemize}
		\item L-System: gera��o de �rvores e cidades.
		\item Fractais e Perlin Noise: gera��o de terrenos e texturas
	\end{itemize}
\end{itemize}

\end{frame}


\begin{frame}\frametitle{O que � gera��o procedural?}
Tabela com n�mero de entradas X n�vel de gera��o procedural.
\end{frame}



\begin{frame}\frametitle{Vantagens da gera��o procedural} 
\begin{itemize}
	\item Flexibilidade: alterando os par�metros do algoritmo, � poss�vel gerar um grande n�mero de modelos.
	\item Espa�o: n�o h� necessidade de um grande espa�o em disco, j� que tudo ser� ditado por algoritmos.
\end{itemize}
\end{frame}

\begin{frame}\frametitle{Exemplos}
\begin{columns}
	\begin{column}{5cm}
	\begin{itemize}
		\item \alert<1>{.kkrieger\\} \only<1>{\scriptsize Praticamente tudo gerado proceduralmente}
		\item \alert<2>{Spore\\} \only<2>{\scriptsize Planetas gerados proceduralmente.}
		\item \alert<3>{SpeedTree\\} \only<3>{\scriptsize �rvores geradas proceduralmente.}
	\end{itemize}
	\vspace{3cm} 
	\end{column}
	\begin{column}{5cm}
	\begin{overprint}
		\includegraphics<1>[width=1.0\linewidth]{img/kkrieger}
		\includegraphics<3>[width=1.0\linewidth]{img/speedtree}
	\end{overprint}
	\end{column}
\end{columns}
\end{frame}

