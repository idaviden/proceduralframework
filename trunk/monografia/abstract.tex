\begin{abstract}

With the increasing demand for 3D models in areas such as games, the procedural content generation is becoming increasingly popular, in order to reduce costs and development time. With procedural techniques, it is possible to create a large number of models modifying a few parameters on the algorithms responsable for the generation. However, procedural techniques have a low user interaction, creating models that are not always expressive. This work focus is on procedural generation of terrains and it proposes a framework that increases the user participation level, in order to create models that better translate his expectations. Well know procedural techniques are explored and mixed with terrain and models inserted by the user.

    \textbf{Keywords}: Procedural content generation, terrain.
\end{abstract}
